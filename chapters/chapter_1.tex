
% The \phantomsection command is needed to create a link to a place in the document that is not a
% figure, equation, table, section, subsection, chapter, etc.
% https://tex.stackexchange.com/questions/44088/when-do-i-need-to-invoke-phantomsection
\phantomsection

\chapter{Integer Programming}\label{chap:integer-programming}


Integer programming, a subset of mathematical programming, addresses optimization problems where decision variables are required to take on integer values.
Specifically, mixed-integer linear programming (MILP) extends this concept by encompassing the assumption that for each possible discrete decision, a (continuous) linear program has to be solved.
The complexity of MILP problems often necessitates sophisticated solution methods to find optimal or near-optimal solutions.
This chapter provides an overview of MILP problem-solving techniques, ranging from exact methods like the branch-and-bound algorithm to approaches to provide approximate solutions, such as heuristics and matheuristics.
By delving into these methodologies, we lay the groundwork for the subsequent discussion on deep learning-based primal heuristics, which aim to enhance the efficiency of MILP problem solving.

% \section{Mathematical Programming}
\section{Integer and Combinatorial Optimization}

A solution for an integer and combinatorial optimization problem is the maximum or minimum value of a multivariate function that respects a series of inequality and equality constraints and integrality restrictions on some or all variables~\cite{nemhauserIntegerCombinatorialOptimization1999}.
Mathematical programming is a language naturally suitable to formulate integer and combinatorial optimization problems, for example, in the form
\begin{equation}\label{eq:general-ip}
    \begin{split}
	\min_{\bm{x}} \quad & f\left( \bm{x} \right) \\
	\textrm{s.t.} \quad & \bm{h}\left( \bm{x} \right) = \bm{0} \\
	  & \bm{g}\left( \bm{x} \right) \le \bm{0} \\
	  & \bm{x} \in \Z^{n}\times \R^{p}
    ,\end{split}
\end{equation}
with $n$ integer variables and $p$ continuous variables.
Note that both $\bm{h}$ and $\bm{g}$ are vector-valued functions and $\bm{0}$ is a null vector of appropriate dimension.
It is not difficult to see that the formulation of \eqref{eq:general-ip} encompasses a wide range of problems.
Examples include train scheduling, airline crew scheduling, production planning, electricity generation planning, and cutting problems~\cite{wolseyIntegerProgramming1998}.

Given a problem formulated as in \eqref{eq:general-ip}, we will write \[
\mathcal{X}=\left\{ \bm{x}\in \Z^{n}\times \R^{p}: \bm{h}\left( \bm{x} \right) = \bm{0}, \bm{g}\left( \bm{x} \right) \le \bm{0}\right\} 
\] as the set of \emph{feasible solutions}.
A feasible solution $\bm{x}^{*}\in \mathcal{X}$ is \emph{optimal} if, and only if, there is no other feasible solution results in a lower value of the \emph{objective function} $f: \Z^{n}\times \R^{p} \longrightarrow \R$, i.e., $\bm{x}^{*}$ is optimal $\iff f(\bm{x}^{*}) \le f(\bm{x}) ,\,\forall \bm{x}\in \mathcal{X}$.



\section{Mixed-Integer Linear Programs}

\section{Solving MILP Problems}

\subsection{The Branch-and-Bound Algorithm}

\subsection{Heuristics}

\subsection{Matheuristics}

