
% The \phantomsection command is needed to create a link to a place in the document that is not a
% figure, equation, table, section, subsection, chapter, etc.
% https://tex.stackexchange.com/questions/44088/when-do-i-need-to-invoke-phantomsection
\phantomsection

\chapter{Experiments}\label{chap:experiments}

To address the objective of this dissertation, experiments are conducted to evaluate learning-based primal heuristics for Mixed-Integer Linear Programming (MILP).
The ONTS problem (see Chap.~\ref{chap:onts}) serves as a realistic application to benchmark the selected techniques.

As discussed in Section~\ref{sec:onts-problem-statement}, during mission execution (with the nanosatellite in orbit), a new schedule must be generated during the communication window.
This involves optimizing multiple instances of the ONTS problem, given varying sets of tasks and updated nanosatellite information.
Each set of tasks is evaluated based on the resulting schedule, in an iterative process of including new tasks until scheduling becomes infeasible.
Therefore, during the communication window, quickly finding a good solution to a problem instance is more crucial than finding an optimal solution.
In other words, an efficient heuristic is crucial to allow for more iterations, which leads to a better set of tasks scheduled for execution.

The remaining of this chapter details the development and the experiments with the proposed learning-based heuristics for the ONTS problem.
This includes data acquisition, model architecture, training, and experiment setup.
Furthermore, the performance of the proposed learning-based heuristics is assessed on realistic instances of the ONTS problem.


\section{Data}

Problem selected = ONTS problem

\subsection{Instance space: the FloripaSat I mission}

The instance space is defined based on the FloripaSat I mission
    plug info

Describe parameter space

\subsection{Data acquisition}

It is assumed that the instances that will be found in practice are uniformly distributed within the range defined by the parameters of FloripaSat I
Therefore U ~ I

As the history of instances solved is not available, we build our dataset...
    continue from paper

\section{Deep Learning Model Architectures}

- GNNs are very promising, considered sota
- FCNs are not suitable for our problem, because it must suit instances with varying number of jobs (and, thus, varying number of variables), while FCNs have a fixed output dimension.

\section{Training}

\section{Evaluation}

