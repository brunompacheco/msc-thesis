
% The \phantomsection command is needed to create a link to a place in the document that is not a
% figure, equation, table, section, subsection, chapter, etc.
% https://tex.stackexchange.com/questions/44088/when-do-i-need-to-invoke-phantomsection
\phantomsection

\chapter{Evaluation of Primal Heuristics}\label{chap:evaluation}

As discussed in Chapter\,\ref{chap:integer-programming}, Section\,\ref{sec:heuristics}, a primal heuristic abides from optimality guarantees to focus on a trade-off between solution quality and computational cost (speed).
As a consequence, two primal heuristics for MILP problems can be compared with respect to how fast and how good they provide solutions, and whether the solutions are feasible or not.
In this chapter, two approaches are discussed to evaluate primal heuristics.
The first (Sec.~\ref{sec:standard-evaluation-metrics}) is based on standard evaluation metrics, therefore, it uses multiple metrics, one for each perspective in which an heuristic can be said superior to another.
The second (Sec.\,\ref{sec:primal-dual-curve}) is specific to primal heuristics that improve the candidate solution over time, such as matheuristics.
By tracing the progress of the candidate solution, it provides a single metric that takes into account the characteristic trade-off of primal heuristics.

% - como discutido na sec XXX, ao abrir mão da garantia de otimalidade, uma heurística primal oferece a possibilidade de trocar qualidade da solução por velocidade de otimização.
% - nesse sentido, duas heurísticas podem ser comparadas em função dessa relação
% - neste capítulo, duas abordagens serão apresentadas para caracterizar a qualidade de heurísticas. a primeira, através de múltiplas métricas que são comumente para avaliar qualquer tipo de heurística
% - a segunda, específica para matheuristics, que explora a sua característica de aprimorar a solução candidata ao longo do tempo

\section{Standard Evaluation Metrics}\label{sec:standard-evaluation-metrics}

\begin{itemize}
    \item Runtime
    \item GAP
    \item Infeasibility\%
\end{itemize}

\section{The Primal-dual Curve}\label{sec:primal-dual-curve}

% see https://www.ecole.ai/2021/ml4co-competition/#metrics

- add illustration of the backwards-filling scheme for fairness

