%% intro.tex
%%
%% Copyright 2017 Evandro Coan
%% Copyright 2012-2016 by abnTeX2 group at http://www.abntex.net.br/
%%
%% This work may be distributed and/or modified under the
%% conditions of the LaTeX Project Public License, either version 1.3
%% of this license or (at your option) any later version.
%% The latest version of this license is in
%%   http://www.latex-project.org/lppl.txt
%% and version 1.3 or later is part of all distributions of LaTeX
%% version 2005/12/01 or later.
%%
%% This work has the LPPL maintenance status `maintained'.
%% The Current Maintainer of this work is the Evandro Coan.
%%
%% The last Maintainer of this work was the abnTeX2 team, led
%% by Lauro César Araujo. Further information are available on
%% https://www.abntex.net.br/
%%
%% This work consists of a bunch of files. But originally there ware 3 files
%% which are renamed as follows:
%% Renamed the `abntex2-modelo-include-comandos` to `chapters/chapter_1.tex`
%% Renamed the `abntex2-modelo-trabalho-academico.tex` to `chapters/intro.tex`
%% Renamed the `abntex2-modelo-references.bib` to `aftertext/modelo-ufsc-references.bib`
%%
%% This file was originally the main template file, however this main file was
%% split into several new files, which are respectively drastically changed,
%% except this files which contains most of the main documentation message.
%%

% ------------------------------------------------------------------------
% ------------------------------------------------------------------------
% abnTeX2: Modelo de Trabalho Academico (tese de doutorado, dissertacao de
% mestrado e trabalhos monograficos em geral) em conformidade com
% ABNT NBR 14724:2011: Informacao e documentacao - Trabalhos academicos -
% Apresentacao
% ------------------------------------------------------------------------
% ------------------------------------------------------------------------

% The \phantomsection command is needed to create a link to a place in the document that is not a
% figure, equation, table, section, subsection, chapter, etc.
% https://tex.stackexchange.com/questions/44088/when-do-i-need-to-invoke-phantomsection
\phantomsection

% https://tex.stackexchange.com/questions/5076/is-it-possible-to-keep-my-translation-together-with-original-text
\chapter{\lang{Introduction}{Introdução}}\label{chap:intro}
\phantomsection

In the realm of mathematical optimization, Mixed-Integer Linear Programming (MILP) stands as a powerful tool for addressing a wide array of complex decision-making problems.
These problems, prevalent in fields ranging from operations research to finance and logistics, often involve the need to make discrete decisions within a linear framework.
Despite their significance, solving MILP instances efficiently remains a formidable challenge, as the search space expands exponentially with the number of integer variables.
In other words, the combinatorial nature of MILP implies that algorithms with optimality guarantees have intractable running times.

Primal heuristics, which aim to quickly find high-quality feasible solutions to MILP problems, play a crucial role in enhancing the efficiency of optimization algorithms.
Traditional primal heuristics are often rule-based and designed to exploit certain structures of an MILP problem.
As a consequence, they lack adaptability, struggling to generalize across diverse problem instances.
As the landscape of optimization problems continues to evolve, there is a growing need for intelligent and flexible heuristics that can adapt to the intricacies of different MILP instances.

Recently, machine learning techniques have been successfully applied to MILP problems, resulting in effective heuristics.
In contrast to handcrafted heuristics, which rely on expert knowledge to exploit theoretical structures of problem formulations, machine learning aims to identifying the hidden patterns of problem instances.
This data-driven approach relies on the assumption that problem instances are drawn from underlying distributions, and, thus, share characteristics not evident in the mathematical formulation.
Such assumption often holds for practical situations, in which problems must be solved repeatedly and the parameters that define the instances are random variables with unknown distributions.

It is intuitive that a learning-based primal heuristic is built using a solution prediction model.
In fact, the model by itself could be a primal heuristic, as it ideally provides solutions to problem instances.
However, in practice, the probabilistic nature of the learning algorithms 


This master thesis delves into the intersection of deep learning and MILP optimization, exploring the potential of leveraging neural networks to develop data-driven primal heuristics.
The focus lies on primal heuristics built from models trained with supervised learning to predict candidate solutions to MILP problem instances.


The focus lies on the creation of supervised solution prediction models, where the deep-learning architecture is trained on historical data to predict high-quality feasible solutions for MILP instances.
This novel approach aims to overcome the limitations of traditional heuristics by harnessing the power of artificial intelligence to learn intricate patterns and relationships within the problem space.

The overarching goal of this research is to contribute to the development of efficient and adaptable primal heuristics that can significantly improve the performance of MILP solvers.
By combining the strengths of deep learning and optimization, this thesis seeks to advance the state-of-the-art in tackling the inherent challenges posed by MILP problems, ultimately paving the way for more effective decision-making in complex real-world scenarios.


Main objective: to evaluate primal heuristics for MILP problems based on solution prediction models trained with supervised learning techniques

Three research questions:
- how to design deep learning models able to provide candidate solutions for instances of an MILP problem?
- how to incorporate solution prediction models in primal heuristics?
- which supervised learning techniques are most effective to train solution prediction models for primal heuristics?

Objectives:
- Assess the literature on supervised learning solution prediction models for MILP problems, including model architectures, supervised learning algorithms, and learning-based primal heuristics
- Evaluate the most promising techniques in a realistic application with respect to the performance of the resulting primal heuristic

\section{Objectives}\label{chap:objectives}


