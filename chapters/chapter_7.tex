
% The \phantomsection command is needed to create a link to a place in the document that is not a
% figure, equation, table, section, subsection, chapter, etc.
% https://tex.stackexchange.com/questions/44088/when-do-i-need-to-invoke-phantomsection
\phantomsection

% ---
\chapter{Discussion}
\phantomsection

Data acquisition for learning-based heuristics is a particularly challenging task. 
Historical data is seldom available in the volume necessary to compose a training set suitable for modern deep learning techniques, which leads practitioners to resort to data generation~\cite{bengioMachineLearningCombinatorial2021}.
Generating instances, by itself, is not usually a problem, as parameter ranges can be defined with enough margin to encompass values encountered in practice.
However, the solution to these randomly sampled instances is needed.
On top of that, the interest in learning-based heuristic solutions is directly related to the problem difficulty, which, in turn, increases the cost for data generation.
In other words, the bigger the potential for learning-based heuristics, the more expensive it is to acquire training data.

- the "trick" of limiting the time to find an optimal solution, as was done here, is known to impose limits in the generalization of the results.
- As discussed in Yehuda et al., sampling instances from NP-hard problems that are solvable in tractable time is actually equivalent to sampling from an easier sub-problem (cite Cappart's review on GNNs for CO). 
- This, however, highlights the generalization capabilities of GNNs as shown in this work's results, indicating that easier instances can be used to develop solution predictions models capable of tackling hard instances.
- Gasse, 2019, has already shown this generalization capacity to instances larger than seen during training

