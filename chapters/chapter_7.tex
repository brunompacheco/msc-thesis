
% The \phantomsection command is needed to create a link to a place in the document that is not a
% figure, equation, table, section, subsection, chapter, etc.
% https://tex.stackexchange.com/questions/44088/when-do-i-need-to-invoke-phantomsection
\phantomsection

% ---
\chapter{Discussion}\label{chap:discussion}
\phantomsection

The experiments described in Chapter~\ref{chap:experiments} aimed at evaluating the effectiveness of learning-based heuristics in a realistic application, namely, the ONTS problem.
The problem setup involves finding the best set of tasks that results in a high-quality, feasible schedule, at every communication window.
This translates into solving multiple instances of MILP problems in a small window of time.
As a consequence, the NP-hard nature of MILP makes the algorithmic approach to solving such instances impossible.
In this case, the baseline is to run an MILP solver with limited time, which has no guarantees of finding feasible or optimal solutions\footnote{And, thus, can be said a heuristic solution approach.}.

Although all experiments were performed using data from the ONTS problem, the setup is very general, which renders the results relevant to many different applications.
More specifically, the general problem setup is that of repeatedly solving instances of an optimization that follow an unknown distribution, under limited time.
This setup appears, e.g., in the management of energy distribution networks, vehicle routing under varying traffic conditions, workload apportioning across workers, and maritime inventory routing~\cite{gasseMachineLearningCombinatorial2022,papageorgiouMIRPLibLibraryMaritime2014}.
In other words, the solution approach evaluated in the presented experiments is of interest by many different application areas.

A shared challenge across applications of learning-based heuristics is that of data acquisition.
Historical data is seldom available in the volume necessary to compose a training set suitable for modern deep learning techniques, which leads practitioners to resort to data generation~\cite{bengioMachineLearningCombinatorial2021}.
Generating instances, by itself, is not usually a problem, as parameter ranges can be defined with enough margin to encompass values encountered in practice.
However, the solution to these randomly sampled instances is needed.
On top of that, the interest in learning-based heuristic solutions is directly related to the problem difficulty, which, in turn, increases the cost for data generation.
In other words, the bigger the potential for learning-based heuristics, the more expensive it is to acquire training data.

- the "trick" of limiting the time to find an optimal solution, as was done here, is known to impose limits in the generalization of the results.
- As discussed in Yehuda et al., sampling instances from NP-hard problems that are solvable in tractable time is actually equivalent to sampling from an easier sub-problem (cite Cappart's review on GNNs for CO). 
- This, however, highlights the generalization capabilities of GNNs as shown in this work's results, indicating that easier instances can be used to develop solution predictions models capable of tackling hard instances.
- Gasse, 2019, has already shown this generalization capacity to instances larger than seen during training

