
% resumo em português
\setlength{\absparsep}{18pt} % ajusta o espaçamento dos parágrafos do resumo

% resumo em inglês
\begin{resumo}[Abstract]

\noindent

Mixed-Integer Linear Programming (MILP) is a crucial tool for solving complex decision-making problems due to its ability to model combinatorial optimization tasks and arbitrarily approximate nonlinear features.
Deep-learning-based primal heuristics offer a promising solution for efficiently addressing MILP problems.
Focusing on supervised solution prediction models, this dissertation investigates the design, training, and integration of deep learning models into primal heuristics using the Offline Nanosatellite Task Scheduling (ONTS) problem as a test case.
Key findings are drawn on model architecture, loss functions, data acquisition, and meta-heuristic.
On top of that, the proposed learning-based heuristic approaches were able to provide, on one hand, a 35\% reduction in the expected time to find a feasible solution to the ONTS problem, and on another, a 43\% expected gain in the normalized quality of the heuristic solutions.
These results highlight the potential of deep learning approaches to enhance the adaptability and efficiency of optimization solutions, with future research needed to further explore Graph Neural Network (GNN) generalization and improve data generation techniques.

\vspace{\onelineskip}

\noindent 

\textbf{Keywords}: MILP, Matheuristics, Deep Learning, Learning-based Heuristics, Graph Neural Networks, Nanosatellite Task Scheduling.

\end{resumo}


% resumo em português

\begin{resumo}[Resumo]
\begin{otherlanguage*}{portuguese}

\noindent

A programação linear inteira mista (\emph{Mixed-Integer Linear Programming}, MILP) é crucial no auxílio à tomada de decisão em cenários complexos devido à sua capacidade de modelar problemas de otimização combinatória e aproximar dinâmicas não-lineares.
Heurísticas baseadas em modelos de aprendizagem profunda (\emph{deep learning}) oferecem uma solução promissora para resolver problemas MILP eficientemente.
Tendo foco em modelos supervisionados para predição de soluções, esta dissertação investiga o projeto, o treinamento e a integração de modelos de aprendizagem profunda em heurísticas primais, usando o agendamento \emph{offline} de tarefas em nanossatélites (\emph{Offline Nanosatellite Task Scheduling}, ONTS) como um caso de teste.
As principais conclusões deste trabalho se referem à arquitetura dos modelos, às funções de perda, à aquisição de dados e à meta-heurísticas.
Além disso, as heurísticas baseadas em aprendizagem propostas para o ONTS foram capazes de reduzir, em média, 35\% do tempo necessário para encontrar uma solução factível, e um ganho médio de 43\% na qualidade das soluções encontradas.
Esses resultados destacam o potencial da aprendizagem profunda em gerar heurísticas adaptáveis e eficientes para problemas de otimização, direcionando pesquisas futuras para a investigação da capacidade de generalização de redes neurais baseadas em grafos e de técnicas para geração de dados sintéticos.

\vspace{\onelineskip}

\noindent
\textbf{Palavras-chaves}: MILP, Matheuristics, Deep Learning, Learning-based Heuristics, Graph Neural Networks, Nanosatellite Task Scheduling.

\end{otherlanguage*}
\end{resumo}


% resumo expandido (obrigatório para trabalhos em inglês)
\begin{resumo}[Resumo Expandido]
\begin{otherlanguage*}{portuguese}

\noindent\textbf{\large Introdução}\newline
{\noindent}

Esta dissertação explora a aplicação de heurísticas primais baseadas em modelos de aprendizagem profunda à programação linear inteira mista (\emph{Mixed-Integer Linear Programming}, MILP).
MILP é uma ferramenta chave da pesquisa operacional devido a sua capacidade de modelar problemas combinatórios e de aproximar, com precisão arbitrária, dinâmicas não-lineares.
Além disso, existência de \emph{softwares} bem-estabelecidos para resolver problemas de MILP facilita a sua aplicação fácil e a torna confiável.

Encontrar soluções ótimas para problemas de MILP de forma eficiente é um desafio devido ao crescimento exponencial do espaço de busca em função do número de variáveis inteiras.
Como consequência, heurísticas primais se tornam valiosas como uma forma tratável de encontrar soluções de boa qualidade em contextos de recursos limitados.
Entretanto, projetar uma heurística primal efetiva é uma tarefa que requer um grande esforço de engenharia pois deve ser feita sob medida para o problema alvo.
Recentemente, técnicas de aprendizagem profunda foram propostas para criar heurísticas especializadas de forma automática, explorando padrões existentes nos dados do problema alvo.


\noindent\textbf{\large Objetivos}\newline
{\noindent}

O principal objetivo desta dissertação é avaliar heurísticas primais para problemas de MILP baseadas em modelos de predição de solução treinados com supervisão.
Este objetivo é subdividido em três:
\begin{itemize}
    \item Analisar a literatura de aprendizado supervisionado para modelos de predição de solução de problemas de MILP, incluindo arquiteturas, algorithmos de aprendizagem e heurísticas primais baseadas em aprendizagem; 
    \item Desenvolver heurísticas primais baseadas em aprendizagem para uma aplicação realista, incluindo as técnicas mais promissoras encontradas na literatura; e
    \item Avaliar as técnicas para heurísticas baseadas em aprendizagem com respeito a performance empírica na aplicação selecionada e as garantias teóricas fornecidas por cada uma delas.
\end{itemize}

\noindent\textbf{\large Metodologia}\newline
{\noindent} 

Este trabalho avaliou diversas técnicas encontradas na literatura para os mais distintos componentes de modelos de solução de predição para problemas de MILP.
Em relação à arquitetura dos modelos de predição de solução, foram investigados o uso de redes neurais baseadas em grafos (\emph{Graph Neural Networks}, GNNs) com convoluções baseadas no operador SAGE e o compartilhamento de parâmetros da rede entre as suas convoluções.
Duas técnicas distintas de treinamento foram implementadas e avaliadas: a primeira utilizando uma solução (quasi-)ótima, e a segunda utilizando múltiplas soluções para cada instância disponível do problema de otimização.
Técnicas para aquisição de dados também foram analisadas, em particular no contexto da ausência de dados históricos.

Os experimentos foram projetados para analisar a efetividade das heurísticas propostas para o agendamento \emph{offline} de tarefas em nanossatélites (\emph{Offline Nanosatellite Task Scheduling}, ONTS).
O cenário analisado do problema em questão é o agendamento durante a operação do satélite em órbita, que requer a solução de múltiplas instâncias do problema de MILP em uma janela de tempo limitada.
Três diferentes heurísticas baseadas em modelos de solução de predição foram avaliadas tendo dois objetivos distintos: encontrar soluções factíveis no menor tempo possível, e encontrar a melhor solução em um tempo limitado.


\noindent\textbf{\large Resultados e Discussão}\newline

Os experimentos no problema de ONTS indicaram as melhores configurações para modelos de predição de solução.
Em particular, eles apontam uma superioridade do operador SAGE~\cite{hamiltonInductiveRepresentationLearning2017} em relação à convolução original, proposta por \citeonline{kipfSemiSupervisedClassificationGraph2017}, além de um ganho de desempenho ao compartilhar os parâmetros entre as convoluções.
Os melhores modelos de predição de solução foram treinados utilizando múltiplas soluções por instância do problema como supervisão.

A construção de heurísticas primais com os modelos treinados se mostrou mais efetiva quando se dava através da fixação de variáveis binárias através da predição dos modelos (\emph{early-fixing}).
Essa estratégia resultou em heurísticas que reduziram, em média, 35\% do tempo necessário para encontrar uma solução factível, e aumentaram em 43\% a qualidade da solução encontrada dado um tempo limitado de 2 minutos.

A aquisição de dados se mostrou um desafio devido a ausência de dados históricos e ao alto custo para encontrar soluções para as instâncias sintéticas do problema.
De toda forma, a capacidade de generalização dos modelos de predição de solução construídos com GNNs permitiu o treinamento com instâncias mais fáceis (e, portanto, menos custosas) do que aquelas utilizadas para avaliação.


\noindent\textbf{\large Considerações Finais}\newline

Esta dissertação demonstra que heurísticas primais baseadas em aprendizagem profunda são promissoras frente aos desafios da MILP.
Os resultados contribuem para a área de pesquisa de aprendizado de máquina para otimização combinatória ao oferecer uma análise das técnicas mais relevantes encontradas na literatura e uma comparação empírica e não-enviesada em uma aplicação representativa.
Além disso, esta dissertação aponta para uma investigação mais aprofundada sobre os limites da capacidade de generalização das GNNs em problemas de MILP, técnicas de geração de dados sintéticos para modelos de predição de solução, e um refinamento da eficiência desses mesmos modelos em relação aos dados necessário e o desempenho esperado.

\vspace{\onelineskip}

\textbf{Palavras-chaves}: MILP, Matheuristics, Deep Learning, Learning-based Heuristics, Graph Neural Networks, Nanosatellite Task Scheduling.

\end{otherlanguage*}
\end{resumo}

