

% \addtotextpreliminarycontent{\lang{Acknowledgement}{Agradecimentos}}

\begin{agradecimentos}[Agradecimentos]

% TODO: agradecimentos
Esta dissertação, enquanto requisito conclusivo do mestrado em engenharia de automação e sistemas, representa dois anos de dedicação ao estudo e à pesquisa.
% Ainda mais que em meu trabalho de conclusão do curso de engenharia de controle e automação, quaisquer louros colhidos me colocam em dívida 
Tal qual em meu trabalho de conclusão do curso de engenharia de controle e automação, mas com ainda mais veemência, quaisquer louros que eu possa ter colhido são justamente devidos aos mestres, àqueles que me ensinaram ao longo desta jornada na academia.
Aqui, agradeço especialmente ao professor Eduardo Camponogara, meu orientador, que além de um excelente representante desse grupo, também me apoiou e forneceu suporte em múltiplos âmbitos, assim tornando fértil o que eu considero ter sido um período de muito amadurecimento.

Não seria justo deixar de agradecer também a todos os meus colegas do GOS (grupo de pesquisa em otimização de sistemas).
Exemplarmente, agradeço ao professor Laio Oriel Seman, meu co-orientador, pelas tantas ideias e desafios propostos, como também pelo pioneirismo em nosso grupo no que se refere à linha de pesquisa na qual minha dissertação se situa.

Finalmente, agradeço também aos meus, que justificam e motivam tanto a minha dedicação, quanto o meu descanso.
Em particular, sou grato àqueles que estiveram mais próximos - meus primos, meus sogros e, especialmente, minha companheira -, a quem eu credito a instauração do meu sentimento de pertencimento nesta cidade.

\end{agradecimentos}


%Mesmo padrão da seção primária, porém sem indicativo numérico. Assim como: Dedicatória, Resumo, Abstract, Sumário, Listas, Referências, Apêndices e Anexos.
%
%
%Corpo do texto, fonte 10,5, justificado, recuo especial da primeira linha de 1 cm, espaçamento simples.
%
